\secput{analysis}{The Analysis of Complexity Bound}

\subsecput{stage1Analysis}{The complexity bound of stage $1$}
From \thmref{totalNodes} we can see that if we assume the {DAG} has $M$
nodes, the algorithm to construct the {DAG} is no more than $M^2$. In
this section, we are going to clarify the relation of $M$ and the number
of input columns $m$.

Intuitively, $m \in [2, 2^g]$, when $m = 2$, $M = 3 \in O(m^2)$, when 
$m = 2^g$, $M = 3^g = O(m^2)$, also, if $M = f(m)$, this $f$ seems an 
monotonically increasing function. However, it turns out that in the 
worst case, $M$ can be an exponential of $m$. Following theorem tries to 
prove it by induction.

\begin{theorem}
If there are $m$ input columns, each of which is of $g$ bit wide. If we
assume the output {DAG} has $M = f(m)$ nodes in total (including the
input $m$ columns), then in the worst case $M = f(m) = O(2^g)$.
\label{thm:worstCaseM}
\end{theorem}

\begin{IEEEproof}
Actually, the worst case comes when the input bit matrix is diagonal 
as follows: 

\[
\begin{pmatrix}
1 & 0 & \cdots & 0 \\
0 & 1 & \cdots & 0 \\
\vdots & \vdots & \ddots & \vdots \\
0 & 0 & \cdots & 1
\end{pmatrix}
\]

If we induct on the number of input columns, we have:

First, for $m = 2$, $M = 3$. 

Second, if assuming that we already constructed a {DAG} from $m$
input columns, $c_0, c_1, \ldots, c_{m-1}$, and bin $0$ has $b_0 = m$
nodes, bin $1$ has $b_1$ nodes, $\ldots$, bin $i$ has $b_i$ nodes. When we
add in one more node $c_m$, let's count how many nodes will be increased
in each bin, and total.

\begin{itemize}

\item for bin $0$, apparently, $b_0' = b_0 + 1$. 

\item for bin $1$, $b_1' = b_1 + b_0$. The new nodes in bin $1$ are of
the form $c_i c_m$, with $c_i \in$ bin $0$. Because it's diagonal matrix,
so the newly add-in node $c_m$ has a $1$ set at different position from
any of existing old nodes, so there is no collision of newly merged nodes
with old nodes.

\item for bin $2$, $b_2' = b_2 + b_1$. Note that the new nodes in bin $2$ 
are of the form $c_i c_j c_m$, with $c_i, c_j \in b_1$.

\item for bin $i$, $b_i' = b_i + b_{i-1}$.

\item Overall, $\sum_i b_i' = 2 \sum_i b_i$, i.e. the summation of number
of nodes in all bins increases exponentially in this case. Because the
number of columns of diagonal matrix is at most $g$, so it has at most
$2^g$ nodes in all bins after the construction.

\end{itemize}
\end{IEEEproof}

\thmref{worstCaseM} is about the complexity of $M$ in the worst case,
following calculation yields the expected number of $M$.

\begin{eqnarray*}
E[bin(0)] & = & m \\
E[bin(1)] & = & {m \choose 2} \cdot \id{Pr}{\{I_{i, j}\}} \quad \mbox{where $I_{i, j} = 1$ if column $i$ and $j$ differs by only $1$ bit, $0$ otherwise} \\
       & = & {m \choose 2} \cdot \frac{{g \choose 1}}{2^g} \\
       & = & O(gm^2 / 2^g) \\
E[bin(2)] & = & {E[bin(1)] \choose 2} \cdot \id{Pr}{\{I_{i, j}\}} \quad \mbox{where $I_{i, j} = 1$ if column $i$ and $j$ differs by only $1$ non-``x'' position, $0$ otherwise } \\
       & = & (\frac{gm^2}{2^g})^2 \cdot \frac{g-1}{2^{g-1}} \\
       & = & O(\frac{g^3 m^4}{2^{3g}}) \\
\ldots & = & \ldots \\
E[bin(i)] & = & {E[bin(i-1)] \choose 2} \cdot \id{Pr}{\{I_{i, j}\}} \quad \mbox{where $I_{i, j} = 1$ if column $i$ and $j$ differs by only $1$ non-``x'' position, $0$ otherwise } \\
       & = & O(\frac{g^{2i-1} \cdot m^{2^i}}{2^{(2i-1)g}})
\end{eqnarray*}

By non-``x'' position, we distinguish two cases: 

\begin{itemize}

    \item columns $(0, 1, 0)^T$ and $(0, 1, x)^T$ differ by a ``x'' position.

    \item columns $(0, 0, 0)^T$ and $(0, 1, x)^T$ differ by a non-``x'' position.
\end{itemize}

So, $E[\sum_{i=0}^{g} |bin(i)|] = O(\frac{g^{2g} \cdot m^{2^g}}{2^{2g^2}})$

\subsecput{stage2Analysis}{The complexity bound of stage $2$}
Apparently, in recurrence~\ref{eq:recG}, operation ``$\mbox{LCA}$'' has
complexity $O(m)$, set operation ``$-$'' and ``$\cap$'' has complexity
$O(m)$, ``$\oplus$'' has complexity $O(m)$. The recurrence~\ref{eq:recG}
takes a minimum of four terms, each of which has complexity $O(qm)$. If
we assume there are total $M$ nodes in {DAG} $T$, in total, the complexity
is $O(M \sum_{q = 1}^{k} qm) = O(k^2 M m)$.

\punt{%
\subsecput{totalComplexity}{Overall complexity of the algorithm}
From stage $1$, we have $M = O(m^2)$. From stage $2$, we have complexity
$O(k^2 M m)$. So combine these two, we have overall complexity $O(k^2 m^3)$
for the algorithm.
} % end punt
