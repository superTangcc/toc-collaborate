% -*- Mode: LaTeX -*-
\secput{future}{Future Work}

Some possible future work may be:

\begin{itemize}
    \item How to relax the current constraints on polygonal range queries
    in $2$-D grids, such as arbitrary triangular queries.
    \item How to extend the polygonal range query in $2$-D to polytopal 
    range query in $3$-D grids.
\end{itemize}

\punt{In this paper, we developed a data reduction method, developed a
meta-algorithm $M$ for the partial sum query problem over $d$-dimensional
array ($d \geq 1$) of size $N$. Provided the partial sum operation is
valid over a semi-group and is associative and commutative, by taking
a user's input preprocessing algorithm $P$ of complexity $\Theta(N
\log^p N)$ in both space and time, where $0 \leq p \leq d$, and query
algorithm $Q$ of complexity $q(N, d)$, the meta-algorithm will output
a preprocessing algortihm $M(P)$ of complexity $\Theta(N \log^{p-1}
N)$,  a query algorithm $M(Q)$ with complexity $\Theta(2^{2d \cdot
\log^{*} N} \cdot q(N, d))$. Moreover, since the meta-algorithm treats
the input algorithm as a black box, if we view the meta-algorithm as a
higher-order function, by applying $M$ to $P$ at most $p$ times, i.e.
$\underbrace{M(M(\ldots M}_{p}(P)))$, we can reduce the preprocessing
complexity to optimal ($\Theta(N)$), with near-constant factor penalty on
query time (which can possibly be improved either by employing a similar
trick in \cite{AmirFiLe07} or by exploiting the internal structure of
the input query algorithm).  In addition, the result algorithm produces
good memory locality with an asymptotically smaller cache-oblivious
data structrure.}

