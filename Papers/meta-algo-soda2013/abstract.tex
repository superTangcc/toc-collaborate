\begin{abstract}

Algorithms in the literature for performing range queries on
multidimensional grid typically assume that the ranges are orthogonal.
This paper extends the algorithm to cover non-orthogonal range queries. 
Our algorithm works for
operations drawn from an arbitrary semigroup.  Querying the ``sum'' of
all the grid points within a given $k$-line bounded polygon in a
$2$-dimensional grid with $N$ grid points can be accomplished in
$\Theta(k\alpha^{2}(N))$ time using $\Theta(kN)$ preprocessing space and
time, where $\alpha(N)$ is a functional inverse of Ackermann's
function.  Our algorithm extends A. Yao's one-dimensional
data-reduction method to higher dimensions, as opposed to performing
dimension reduction, as is done by Chazelle and Rosenberg.  

\punt{%
The
preprocessing can be easily parallelized, achieving a parallelism of
$\Theta(kN/\lg N)$ in the work-span model, assuming $k=O(N)$.
} %punt
\punt{%
 In this paper, we generalize the approach to common partial sum
query in semi-groups, which only assumes associativity and commutativity.
In this paper, we consider static partial sum problem, which allows
preprocessing for later online queries.  Yao's paper \cite{Yao82,
Yao85} lay out the foundation for general rectangular-shaped partial
sum query, and achieves an asymptotic bound of $O(N alpha (N))$ in
preprocessing space and time with $O(alpha(N))$ query overhead for 1D
space, where $alpha(N)$ is the inverse Ackermann function.  Chazelle
and Rosenberg \cite{ChazelleRo89} extended it to multi-dimensional
space via dimension reduction method, and achieves $O(N alpha^d (N))$
asymptotic bound in preprocessing time and space with $O(alpha^d (N))$
query overhead, where $d$ stands for the number of dimensions. Later on,
they \cite{ChazelleRo91} also proved that the bound for 1D algorithm
is optimal.  However, we find that just by employing dimension
reduction method only, it is very hard, if not impossible, to handle
irregular-shaped query, such as triangular query, polygonal query in
multi-dimensional space. So we introduce the data reduction method into
the preprocessing stage, which solves arbitrary triangular and even
polygonal (convex or concave) query in general semi-group at the same
asymptotic bound (alpha bound) as rectangular query. The notion of data
reduction results in a much simpler algorithm for range query compared
with original algorithm proposed by Chazelle and Rosenberg. which we call
it meta-algorithm for the reason it accepts as input one algorithm and
output a new algorithm with asymptotic bound improved, We implemented the
algorithm,  for 1D, 2D, and 3D problem, and compared the performance of
initial $O(N \log^d N)$ bound algorithm with the algorithm outputed by the
meta-algorithm, which achieves $O(N (\log^* N)^d)$ bound, $O(N (\log^{**}
N)^d)$ bound, $\ldots$, up to $N (alpha (N))^d$ bound. Not surprisingly,
the performance comparison results match well with their theoretical
asymptotic bound, which proved the efficiency of the meta-algorithm.
Moreover, the algorithm turns out to be straightforwardly parallelizable
due to the simple data dependency pattern. We complement with the
parallelism analysis in work/span model \cite{}. So far, our approach
has the constraint that all slopes of the irregular-shaped query must be
known beforehand for proper preprocessing.  How to solve the same problem
when not all slopes are known in advance is left as an open problem.
} %punt
\end{abstract}

\begin{IEEEkeywords}
polygonal partial sum problem, semi-group, multi-dimensional algorithm,
parallel algorithm, meta-algorithm, higher-order function

\end{IEEEkeywords}
% LocalWords:  LocalWords Pochoir Cilk multicore
