\secput{expr}{Experimental Results}

% experimental results of the meta-algorithm, along with analysis

% In addition to the theoretical work, 
We have implemented our initial algorithm and meta-algorithm (pseudo-codes
in appendix \secref{apdx-init-2D} \secref{apdx-meta-2D}) for the
orthogonal range partial-sum problem on commutative semigroups.
There are empirical studies of the Range Minimum Query (RMQ)
problem \cite{FischerHe06}, which is a special case of the partial-sum
problem. But to the best of our knowledge, ours is the first experimental
study of the static partial-sum problem using an algorithm with
an $\alpha$ bound. We implemented the initial algorithm, which has
a complexity of $O(N \log^d N)$, and compared its performance with a
couple of algorithms generated by our meta-algorithm, that is, an $O(N
(\log^* N)^d)$ algorithm and an $O(N \alpha^d (N))$ algorithm.

We implemented all algorithms in {C++} standard {C++11} \cite{C++11}.
The use ``unsigned long'' as the datatype, and the partial sum operation
was simply the general integer ``+'' operation wrapped in a lambda
function and passed as a function object to all algorithms. All the query
time is calculated as an arithmetic mean of $1000$ random queries over
the entire $d$-dimensional grid.

We ran our experiments on an Intel Core i7 (Nehalem) machine.
~\footnote{Intel Xeon X5650, with clock frequency $2.66$ GHz, $32$KB L1
data cache/core, $256$KB data cache/core, $12$MB L3 cache/socket. Compiler
is icc 12.1.0.} In all performance graphs, the vertical axis is ``problem
size'' in ``unsigned long'', which is $64$-bits long, and the horizontal
axis is ``measured time'' in ``milliseconds''.  So in all performance
graphs, the closer a curve to the horizontal axis, the better the
performance of the corresponding algorithm. To verify the functional
correctness of the meta-algorithm and to compare the query overhead,
we implemented a na\"ive scan through algorithm which scans through all
the points in the query range and sums ($\oplus$) them up to answer any
online query.

From all graphs in \figref{meta-PP}, we see that the preprocessing time
of the $\alpha$ bound algorithm is always better than the $\log^*$ bound
algorithm, which, in turn, is better than the initial algorithm of $\log$
bound. So our experimental results match theoretical predictions, which
confirms the efficiency of the meta-algorithm.

From all graphs in \figref{meta-query}, we see that compared with
na\"ive scan algorithm, the query overhead of our initial algorithm or
those transformed by meta-algorithm is negligible. 
% So, it's really an $O(1)$ algorithm compared with naive scan algorithm.

In the final version, we will include more performance data
on triangular and polygonal preprocessing and query in $2$-D grids.
